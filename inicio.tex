\documentclass[stu, 12pt,letterpaper,donotrepeattitle,floatsintext,natbib]{apa7}
% man, stu
\usepackage[utf8]{inputenc}
\usepackage[spanish]{babel}
\usepackage[normalem]{ulem}
\usepackage{graphicx}
\usepackage{float}
\usepackage{marvosym}
\usepackage{comment}
\usepackage{hyperref} % Enlaces clicables en el índice
\usepackage{lipsum}   % Para texto de ejemplo


% Numeración y niveles del índice
\setcounter{tocdepth}{3}     % Muestra hasta subsubsections en el índice
\setcounter{secnumdepth}{3}  % Numera hasta subsubsections

\title{Título del Proyecto}
\author{Jesús Alonso Vilca Tumbalobos}
\affiliation{Nombre de la Institución: SENATI}
\course{Nombre del Curso}
\professor{Nombre del Profesor}
\duedate{08 de agosto de 2025}

\begin{document}
	
	\maketitle
	
	\tableofcontents %  Este comando genera el índice automáticamente
	\newpage
% --------------------------	
	% CAPÍTULO I
	\section{CAPÍTULO I: Información de la empresa}
	
	\subsection{Razón social}
	
	\begin{description}
		\item[Nombre de empresa:] CESTYS - PERU E.I.R.L.
		\item[RUC:] 20600604733
		\item[Dirección:] AV.Independencia  N°257 
		\item[Gerente General:] GAMION ESPEJO GUDELIA YADHIRA
		\item[Sector:] Privado
		\item[Actividades Económicas:] Principal - 8530 - ENSEÑANZA SUPERIOR
	\end{description}
	\begin{figure}
		\centering
		\includegraphics[width=0.7\linewidth]{Figuras/11-08-08-2025-11}
		\caption{Dirección de la empresa CESTYS - PERU E.I.R.L.}
		\label{fig:11-08-08-2025-11}
	\end{figure}
	

	
	\subsection{Misión, Visión, Objetivos, Valores de la empresa}
	
	
	\subsubsection{Misión}
	Brindar programas de capacitación en Tecnología, Gestión Pública, Ingeniería, Arquitectura y Salud, con una metodología práctica y actualizada, que permita a nuestros estudiantes adquirir las habilidades necesarias para insertarse en el mundo laboral en corto tiempo, impulsando así su desarrollo personal y profesional.
	
	\subsubsection{Visión}
	Ser reconocidos a nivel nacional como una institución líder en formación técnica y profesional de corta duración, que transforma vidas al facilitar el acceso rápido al empleo y promover la continuidad de estudios superiores.
	
	\subsubsection{Objetivo general}
	Formar profesionales competentes en un periodo de 2, 4 o 6 meses, capaces de responder a las demandas del mercado laboral y generar oportunidades para el crecimiento educativo y económico de nuestros egresados.
	
	\subsubsection{Valores}
	\begin{itemize}
		\item \textbf{Compromiso:} Trabajamos con dedicación para garantizar la calidad de la enseñanza y el éxito de nuestros estudiantes.
		\item \textbf{Excelencia:} Mantenemos altos estándares académicos y profesionales.
		\item \textbf{Innovación:} Incorporamos metodologías y contenidos actualizados según las tendencias del mercado.
		\item \textbf{Responsabilidad:} Cumplimos con nuestra labor educativa con ética y transparencia.
		\item \textbf{Trabajo en equipo:} Fomentamos la colaboración entre docentes, personal y alumnos para alcanzar objetivos comunes.
	\end{itemize}
	
	
	
	\subsection{Productos, mercado, clientes}
	\subsubsection{Productos}
	Los productos de esta institución educativa son servicios de formación y capacitación profesional, no objetos físicos. La oferta académica está diseñada para brindar conocimientos y habilidades relevantes, permitiendo a los estudiantes mejorar su perfil profesional y personal. La empresa se dedica a enseñar y capacitar en diversas áreas de alta demanda.
	
	\begin{itemize}
		\item \textbf{Programas de capacitación:}
		
		\begin{itemize}
			\item Administración y gestión empresarial (Cajero Financiero y Comercial, Contabilidad y Finanzas, etc).
			\item Ciencias de la salud (Asistente de farmacia, Asistente dental, Asistente en fisioterapia y rehabilitación).
			\item Gestión pública (Administración pública, políticas públicas).
			\item Ingeniería y arquitectura (Topografía, Lectura de planos, etc).
			\item Tecnología e informática ( Ofimatica, Edicion y postproduccion de videos, etc).
		\end{itemize}
		
		
		\item \textbf{Modalidades de estudio:} Presencial y virtual, ofreciendo flexibilidad y accesibilidad para diferentes perfiles de estudiantes.
	\end{itemize}
	
	\subsubsection{Mercado}
	El mercado para esta institución está determinado por factores como ubicación, demanda y competencia.
	
	\begin{itemize}
		\item \textbf{Ubicación geográfica:} Principalmente en Ayacucho, con alcance nacional e internacional gracias a la modalidad virtual.
		\item \textbf{Tamaño del mercado:} Jóvenes de 17 a 25 años y profesionales que buscan formación continua.
		\item \textbf{Tendencias:} Aumento de la demanda de educación virtual y especializada, impulsada por la digitalización y necesidades del mercado laboral.
		\item \textbf{Competencia:} Universidades e institutos (públicos y privados), que compiten en calidad académica, infraestructura, costos y prestigio.
	\end{itemize}
	
	\subsubsection{Clientes}
	Los principales clientes son los estudiantes, quienes se agrupan en dos grandes segmentos:
	
	\paragraph{Estudiantes de pregrado}
	\begin{itemize}
		\item \textbf{Demografía:} Jóvenes recién egresados de secundaria (17 a 25 años).
		\item \textbf{Motivación:} Obtener un título profesional, mejorar su calidad de vida y contribuir a su comunidad.
		\item \textbf{Necesidades:} Educación de calidad, tecnología, apoyo académico, orientación, costos accesibles, empleabilidad.
	\end{itemize}
	
	\paragraph{Estudiantes de posgrado y educación continua}
	\begin{itemize}
		\item \textbf{Demografía:} Profesionales mayores de 25 años, activos en el mercado laboral.
		\item \textbf{Motivación:} Especializarse, ascender, cambiar de carrera o actualizar conocimientos.
		\item \textbf{Necesidades:} Flexibilidad, modalidad virtual, contenido actualizado, docentes con experiencia, red de contactos.
	\end{itemize}
	
	\subsection{Estructura de la Organización}
	\subsection{Otra información relevante de la empresa donde se desarrolla el proyecto}
	
	% CAPÍTULO II
	\section{CAPÍTULO II: Fundamentación del proyecto}
	\subsection{Identificación del problema técnico en la empresa}
	\subsection{Objetivos del Proyecto de Innovación y/o Mejora}
	\subsection{Antecedentes del Proyecto de Innovación y/o Mejora}
	\subsection{Justificación del Proyecto de Innovación y/o Mejora}
	\subsection{Marco Teórico y Conceptual}
	\subsubsection{Fundamento teórico del Proyecto de Innovación y Mejora}
	\subsubsection{Conceptos y términos utilizados}
	
	% --- Capítulo III ---
	
	
	% --- Referencias y Anexos ---
	\section{Referencias Bibliográficas}
	\addcontentsline{toc}{section}{Referencias Bibliográficas}
	
	\section{Anexos}
	\addcontentsline{toc}{section}{Anexos}
	
\end{document}
