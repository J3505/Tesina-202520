\addcontentsline{toc}{section}{CAPÍTULO II}
\section{Fundamentación del proyecto}

\subsection{Identificación del problema técnico en la empresa}
El principal problema técnico de la empresa se fundamenta en la baja demanda de sus servicios de capacitación y formación profesional en el mercado de Ayacucho. Si bien la institución ofrece una variedad de programas en áreas de alta demanda (como administración, salud e ingeniería), la respuesta del mercado local no ha sido la esperada.

Este problema no se limita solo a la falta de inscripciones, sino que se manifiesta en varios aspectos técnicos y estratégicos:

\begin{itemize}
	\item \textbf{Falta de penetración en el mercado local:} La institución no ha logrado posicionarse como una opción de referencia para la capacitación en Ayacucho. Esto puede deberse a una limitada estrategia de marketing, una percepción desfavorable de la calidad de los cursos o a una oferta que no se ajusta completamente a las necesidades específicas de los profesionales y empresas locales.
	
	\item \textbf{Dependencia excesiva del mercado presencial:} La baja demanda en la modalidad presencial demuestra una vulnerabilidad de la empresa, ya que su modelo de negocio está fuertemente ligado a un mercado geográfico limitado. Si el mercado local no tiene la capacidad de absorber la oferta, es necesario buscar alternativas para su crecimiento.
	
	\item \textbf{Ineficaz aprovechamiento de la modalidad virtual:} La modalidad virtual, que podría ser la solución a la baja demanda local, no está siendo utilizada de manera efectiva. Esto puede deberse a una deficiente estrategia de marketing digital, una plataforma tecnológica poco atractiva o un desconocimiento del público objetivo fuera de Ayacucho.
\end{itemize}

\subsection{Objetivos del Proyecto de Innovación}
\subsubsection{Objetivo General}
Desarrollar e implementar una plataforma web de gestión académica y marketing digital para la empresa de capacitación, con el fin de superar la baja demanda local, optimizar la administración de cursos y ampliar el alcance de la oferta educativa a nivel nacional e internacional.
\subsubsection{Objetivos Específicos}
\begin{itemize}
	\item {Optimizar la administración de cursos}
	\item {Mejorar la promoción y el alcance de la oferta}
	\item {Generar datos para la toma de decisiones}
\end{itemize}

\subsubsection{Objetivos Operativos}
\begin{itemize}
	\item {Desarrollo de la plataforma}
	\item {Implementación de la estrategia de contenido}
	\item {Capacitación y lanzamiento}
\end{itemize}


\subsection{Antecedentes del Proyecto de Innovación y/o Mejora}

\subsubsection{Locales}
Proyectos de la DRE Ayacucho y FONDEP: La Dirección Regional de Educación de Ayacucho (DRE) y el Fondo Nacional de Desarrollo de la Educación Peruana (FONDEP) han promovido concursos y ferias de innovación educativa. Estos proyectos, aunque principalmente enfocados en el ámbito escolar, demuestran una creciente apertura en la región hacia la integración de la tecnología para mejorar la calidad educativa \citep{minedu2021}.

Capacitaciones docentes locales: Se han identificado proyectos de capacitación y actualización de estrategias educativas para docentes, como el caso de \textit{Yachachinapaq Pusanakuy}. Este tipo de iniciativas, documentadas en informes técnicos y publicaciones regionales \citep{garcia2020}, revelan una demanda latente por la formación continua y el desarrollo de habilidades profesionales, lo que valida la necesidad de una plataforma que ordene y promueva dicha oferta de manera más accesible.


\subsubsection{Nacionales}
Plataforma Nacional de Talento Digital del Estado Peruano: Esta iniciativa gubernamental, impulsada por la Secretaría de Gobierno Digital, ofrece cursos virtuales para formar a ciudadanos en tecnologías de la información. Su éxito reside en la centralización de la oferta y la accesibilidad, demostrando la eficacia de una plataforma única para atraer a un público masivo \citep{pcm2022}.

Proyectos de Innovación de la UPN y otras universidades privadas: La Universidad Privada del Norte (UPN) y otras instituciones han implementado plataformas de formación continua y proyectos de tesis enfocados en la implementación de plataformas e-learning para la capacitación de colaboradores. Estos proyectos, documentados en estudios como el de \citep{ramos2021}, validan el modelo de negocio y tecnológico propuesto, subrayando la importancia de la flexibilidad y la adaptación a las necesidades del mercado.


\subsubsection{Internacionales}
Coursera es una de las plataformas de cursos en línea más grandes del mundo, ofreciendo programas de universidades de prestigio. 
Su modelo de negocio se basa en un catálogo bien organizado, una interfaz intuitiva y un sistema de certificaciones reconocido. 
Se considera el estándar de oro en cuanto a diseño de experiencia de usuario (UX/UI) y organización de catálogos \citep{gupta2020}.

La Universitat Oberta de Catalunya (UOC) fue una de las primeras universidades 100\% online. 
Su modelo se centra en la gestión de la relación con el estudiante a distancia y en la implementación de una plataforma tecnológica que soporta procesos académicos y administrativos. 
Ofrece un modelo de gestión académica completo diseñado para educación virtual \citep{uoc2023}.

Khan Academy es una organización sin fines de lucro que ofrece una amplia biblioteca de contenido educativo gratuito. 
Su éxito en la creación de contenido de calidad y el uso estratégico de la tecnología son notables. 
Demuestra cómo el marketing de contenidos puede atraer a una audiencia masiva y construir credibilidad \citep{khan2022}.

\subsection{Justificación del Proyecto de Innovación y/o Mejora}
\subsubsection{Justificación social}
El proyecto tiene un impacto social directo al democratizar el acceso a la educación continua y de calidad. Al ofrecer cursos en modalidad virtual, se superan las barreras geográficas y de tiempo, permitiendo a personas que viven en zonas rurales de Ayacucho, o que tienen horarios laborales exigentes, capacitarse y mejorar sus habilidades profesionales.
\begin{itemize}
	\item {Desarrollo de la plataforma}
\end{itemize}
\subsubsection{Justificación económica}
La plataforma web es una inversión estratégica que generará un retorno económico significativo para la empresa. Su justificación económica se basa en:
\begin{itemize}
	\item \textbf{Ampliación del mercado:}La modalidad virtual permite a la empresa llegar a un público más allá de Ayacucho, abriendo las puertas a un mercado nacional e internacional y multiplicando su potencial de ingresos.
\end{itemize}
\begin{itemize}
	\item \textbf{Optimización de costos:}Al automatizar procesos de inscripción y seguimiento de estudiantes a través de la plataforma, se reduce la carga de trabajo del personal administrativo y se minimizan los costos operativos.
\end{itemize}
\subsubsection{Justificación tecnológica}
El proyecto es una respuesta moderna y eficiente a los desafíos del mercado. La justificación tecnológica se centra en estos tres items.
\begin{itemize}
	\item \textbf{Modernización en la infraestructura:} Se migrará de procesos manuales a un sistema digital integrado, lo que mejorará la eficiencia y la seguridad de la información.
\end{itemize}
\begin{itemize}
	\item \textbf{Flexibilidad y escalabilidad:} La plataforma está diseñada para crecer junto con la empresa. Se podrán añadir nuevos cursos, funcionalidades y usuarios de manera sencilla, sin necesidad de reestructurar todo el sistema.
\end{itemize}
\begin{itemize}
	\item \textbf{Posicionamiento competitivo:} Al adoptar una tecnología de vanguardia, la empresa se diferencia de su competencia en el mercado local y se posiciona como una institución moderna e innovadora, atrayendo a una nueva generación de estudiantes que valora la tecnología.
\end{itemize}
\subsection{Marco Teórico y Conceptual}

\subsubsection{Fundamento teórico del Proyecto de Innovación y Mejora}
Este proyecto se enmarca en la Teoría de la Innovación en Servicios, que postula que las empresas pueden mejorar su competitividad a través de la creación de nuevos servicios o la mejora de los existentes mediante la tecnología. La digitalización no solo optimiza procesos, sino que también crea valor para el cliente al ofrecer mayor accesibilidad y personalización. Al desarrollar una plataforma web, la empresa está innovando en la forma de entregar su servicio educativo, pasando de un modelo predominantemente local a uno híbrido y escalable.

\subsubsection{Conceptos y términos utilizados}
Para la correcta ejecución del proyecto de innovación, se utilizarán diversas tecnologías y herramientas clave que formarán la base de la plataforma web.
\begin{itemize}
	\item \textbf{Frontend y Framework de Desarrollo (Angular):}  Es un framework de desarrollo de código abierto, mantenido por Google, utilizado para construir la interfaz de usuario (frontend) de la plataforma web.
\end{itemize}

\begin{itemize}
	\item \textbf{Backend y Base de Datos (Supabase):}  Es una plataforma de "backend como servicio" (BaaS). Funciona como el cerebro del proyecto, encargándose de la gestión de datos, la autenticación de usuarios y la lógica del servidor. Se eligió por su facilidad de uso y por ofrecer una base de datos PostgreSQL robusta.
	
\end{itemize}
\begin{itemize}
	\item \textbf{ Infraestructura y Despliegue (Cloudflare):}  Es una red de distribución de contenidos (CDN) y un servicio de seguridad. Su función en el proyecto es crucial para mejorar el rendimiento y la protección de la plataforma.
	
\end{itemize}

\begin{itemize}
	\item \textbf{ Conexiones y Funcionalidades Adicionales (Librerías para otras conexiones):}  El proyecto utilizará diversas librerías de código abierto para integrar funcionalidades específicas. Esto se refiere a pequeños paquetes de código que se añaden a la aplicación para realizar tareas concretas sin tener que programarlas desde cero.
	\begin{enumerate}
		\item Integración de pagos (pagos en línea) 
	\end{enumerate}

\end{itemize}




